\documentclass{article}
\usepackage[utf8]{inputenc}
\usepackage{geometry}
\usepackage{graphicx}
\usepackage[hidelinks]{hyperref}

\title{Habits of a Successful Project Manager}
\author{Shashank Verma \\ 40217257
\\
\\
\\
SOEN 6841: Software Project Management}
\date{30/10/2023}


\begin{document}

\maketitle
\pagenumbering{gobble}

\newpage
\tableofcontents
\newpage
\pagenumbering{arabic}

\section{Abstract}
The success of project managers is frequently an amalgam of their technical abilities and interpersonal tendencies. By drawing relationships between traditional management theories and the Scout Law, this paper synthesises the fundamental practises of successful project managers. It investigates the significance of interpersonal skills, proactive problem-solving attitudes, and the ability to sustain strong workplace relationships. A survey of the literature, anecdotal evidence from project management workshops, and the author's own professional observations serve as the foundation for the analysis. The goal of the synthesis is to develop a paradigm that can be used across multiple domains to improve project outcomes and leadership effectiveness.

\section{Introduction}
\subsection{Motivation}
In the current global economy, project managers are widely acknowledged as being essential to the success of projects in a range of businesses. Regardless of the business, their strong management abilities are crucial to raising project productivity and efficiency. Despite the advancements in project management tools and processes, the human element—exemplified by the behaviours of project managers—remains a crucial component of project success. Project management is changing, and this shows how important it is to value human skills in addition to technological knowledge. Project management today includes human dynamics, teamwork, and interpersonal communication in addition to technical execution. Project management expertise is becoming more and more in demand as organisations embrace project-centric approaches.

Now more than ever, educational institutions—especially universities—need to adapt to these changes in the business. It is imperative that project management courses be modified to provide a more comprehensive set of skills. This adaptation is a proactive strategy to equip upcoming project managers for the complex needs of a changing work environment, rather than only a reaction to market demands. It is imperative that technical and interpersonal abilities are balanced in project management education, which highlights the need for a paradigm change in the way project management is taught and perceived.

\subsection{Problem Statement}
The primary issue with present project management is the overemphasis on technical abilities, which frequently ignores soft or people skills. Frameworks such as the PMBOK Guide, which emphasise technical skills over soft skills runs the danger of ignoring soft skills that are just as important—like team dynamics, leadership, and communication—also exhibit this mismatch. This argues for a more integrated approach to project management education, emphasising both technical and interpersonal abilities equally. The objective is to produce managers who excel in both the technical and human aspects of project management by matching academic training with industrial needs. This well-rounded approach is essential for producing project managers who can successfully lead projects in the diverse and changing professional environment of today.



\subsection{Objectives}
This case study aims to identify the non-technical capabilities that are essential to project managers' success. By doing this, we seek to enhance project leadership development and training within organisations. By emphasising the significance of soft skills in achieving project success, this study offers practitioners and their teams practical insights that can be utilised to improve project managers' abilities.

\section{Background Material}
\subsection{The Scout Law as a Leadership Model}

Effective leadership styles in project management significantly impact project success. Project leaders must not only coordinate tasks but also inspire and guide their teams. Unfortunately, project management education often prioritizes technical skills over leadership development. Studies on scouting reveal the importance of a balanced leadership approach. Scouting's focus on individual growth within teams aligns with project management requirements like adaptability, decision-making, problem-solving, and ethical judgment. Scouting's emphasis on real-world challenges and teamwork parallels project management needs. Insights from scouting studies show how young individuals develop strategic thinking, communication, and ethical leadership—critical competencies for project managers. Combining technical expertise with leadership skills fosters a collaborative environment vital for achieving collective project goals. Project management education must evolve to incorporate these leadership aspects, drawing from scouting methodologies to better prepare future professionals for project success.

\subsection{Adaptability in the Face of Disruption}
In the realm of successful project management, the capability to manage and adapt to interruptions is pivotal. Few studies from engineering consultancy exemplifies this, showing how a project manager navigated a work environment rife with interruptions, varying from sudden phone calls to scheduled meetings. Interestingly, 43.2\% of these interruptions were beneficial, contributing to goal achievement and enhanced job satisfaction. This highlights a key trait in effective project management: transforming potential disruptions into opportunities.

In the broader context, successful project managers excel in assessing the urgency of interruptions and integrating them into their workflow, without letting the project goals derail. This ability is critical in fast-paced environments where multitasking and adaptability are essential.

\subsection{People Orientation and Interpersonal Dynamics}
This part of the study emphasises the significance interpersonal skills are to effective project leadership. We seek to explore the impact of a people-oriented approach on project success by looking at how project managers interact with teams and stakeholders.




\subsection{Project Management Working Styles}
The inner workings of a project manager's approach to work are explored, with an emphasis on having the ability to multitask and communicate openly. The purpose of the study is to determine how these working methods impact the finished outcomes of their projects.

\subsection{Fundamental Project Management Responsibilities}
The main activities that project managers need to manage are examined, including their capacity to fulfil assignments, care for and aid the individuals they manage, as well as effective cost management.

\subsection{The Scout Law as a Leadership Model}
A noteworthy section of the research draws similarities between the Scout Law's tenets and the behaviours of successful project managers. The objective of this comparison is to identify a possible global norm for project management success and leadership.

% Add more subsections as needed

\section{Methods \& Methodology}
\subsection{Approach to Problem}
The study's methodology integrates knowledge from two primary sources in a qualitative manner. It begins with a thorough analysis of the body of research, which includes both firsthand reports from experienced project managers and several theoretical models that support project management methodologies. Additionally, observational data from project management workshops is incorporated into the analysis. These courses offer a real-world setting where project managers' tactics, attitudes, and methods of operation are explored. The study attempts to develop a thorough knowledge of the non-technical qualities that contribute to successful project management through this integrated qualitative investigation.

\subsection{Analysis Techniques}
The study's main approach for discovering patterns in the actions and routines of effective project managers is subject analysis. Using a synthesised approach, qualitative data is analysed to identify themes and patterns that appear in various narratives and observations. In order to provide validity and depth, the study also conducts a comparative analysis, placing the themes that have been identified against pre-existing frameworks. The Boy Scouts' Law is one such framework that is utilised for this comparison analysis. It acts as a standard to assess how well the project managers' practises adhere to these time-tested values. The research aims to offer a comprehensive knowledge of the characteristics that characterise effective project management leadership through the use of this dual analytical method.


\section{Results Obtained}
\subsection{Conditions}
Under what conditions were these results obtaineD?

\subsection{Constraints}
constraints on the report.

\subsection{Quality}
How is the quality of the case study.

\section{Conclusions and Future Works}
\subsection{Suggested Improvements}
Suggestions for future improvements.

\subsection{Limitations to Solution}
Describe scenarions where the solutions dont work well enough.

\subsection{Applications in Real World}
Benefits in real world.

\subsection{Conclusion}
Short summary.

\section{References}
Use the \textbackslash cite command to insert citations. For example, see \cite{reference1}.

\subsection{Appendix}
external references here

\subsection{Acknowledgements}
ACKS

\bibliographystyle{IEEEtran}
\bibliography{references}

\end{document}
