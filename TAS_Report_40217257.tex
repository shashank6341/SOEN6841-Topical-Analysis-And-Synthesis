\documentclass{article}
\usepackage[utf8]{inputenc}
\usepackage{geometry}
\usepackage{graphicx}
\usepackage[hidelinks]{hyperref}

\title{Habits of a Successful Project Manager}
\author{Shashank Verma \\ 40217257
\\
\\
\\
SOEN 6841: Software Project Management}
\date{30/10/2023}


\begin{document}

\maketitle
\pagenumbering{gobble}

\newpage
\tableofcontents
\newpage
\pagenumbering{arabic}

\section{Abstract}
The success of project managers is frequently an amalgam of their technical abilities and interpersonal tendencies. By drawing relationships between traditional management theories and the Scout Law, this paper synthesises the fundamental practises of successful project managers. It investigates the significance of interpersonal skills, proactive problem-solving attitudes, and the ability to sustain strong workplace relationships. A survey of the literature, anecdotal evidence from project management workshops, and the author's own professional observations serve as the foundation for the analysis. The goal of the synthesis is to develop a paradigm that can be used across multiple domains to improve project outcomes and leadership effectiveness.

\section{Introduction}
\subsection{Motivation}
In the current global economy, project managers are widely acknowledged as being essential to the success of projects in a range of businesses. Regardless of the business, their strong management abilities are crucial to raising project productivity and efficiency. Despite the advancements in project management tools and processes, the human element—exemplified by the behaviours of project managers—remains a crucial component of project success. Project management is changing, and this shows how important it is to value human skills in addition to technological knowledge. Project management today includes human dynamics, teamwork, and interpersonal communication in addition to technical execution. Project management expertise is becoming more and more in demand as organisations embrace project-centric approaches.

Now more than ever, educational institutions—especially universities—need to adapt to these changes in the business. It is imperative that project management courses be modified to provide a more comprehensive set of skills. This adaptation is a proactive strategy to equip upcoming project managers for the complex needs of a changing work environment, rather than only a reaction to market demands. It is imperative that technical and interpersonal abilities are balanced in project management education, which highlights the need for a paradigm change in the way project management is taught and perceived.

\subsection{Problem Statement}
The primary issue with present project management is the overemphasis on technical abilities, which frequently ignores soft or people skills. Frameworks such as the PMBOK Guide, which emphasise technical skills over soft skills runs the danger of ignoring soft skills that are just as important—like team dynamics, leadership, and communication—also exhibit this mismatch. This argues for a more integrated approach to project management education, emphasising both technical and interpersonal abilities equally. The objective is to produce managers who excel in both the technical and human aspects of project management by matching academic training with industrial needs. This well-rounded approach is essential for producing project managers who can successfully lead projects in the diverse and changing professional environment of today.



\subsection{Objectives}
This case study aims to identify the non-technical capabilities that are essential to project managers' success. By doing this, we seek to enhance project leadership development and training within organisations. By emphasising the significance of soft skills in achieving project success, this study offers practitioners and their teams practical insights that can be utilised to improve project managers' abilities.

\section{Background Material}
\subsection{The Scout Law as a Leadership Model}

Effective leadership styles in project management significantly impact project success. Project leaders must not only coordinate tasks but also inspire and guide their teams. Unfortunately, project management education often prioritizes technical skills over leadership development. Studies on scouting reveal the importance of a balanced leadership approach. Scouting's focus on individual growth within teams aligns with project management requirements like adaptability, decision-making, problem-solving, and ethical judgment. Scouting's emphasis on real-world challenges and teamwork parallels project management needs. Insights from scouting studies show how young individuals develop strategic thinking, communication, and ethical leadership—critical competencies for project managers. Combining technical expertise with leadership skills fosters a collaborative environment vital for achieving collective project goals. Project management education must evolve to incorporate these leadership aspects, drawing from scouting methodologies to better prepare future professionals for project success.

\subsection{Adaptability in the Face of Disruption}
In the realm of successful project management, the capability to manage and adapt to interruptions is pivotal. Few studies from engineering consultancy exemplifies this, showing how a project manager navigated a work environment rife with interruptions, varying from sudden phone calls to scheduled meetings. Interestingly, 43.2\% of these interruptions were beneficial, contributing to goal achievement and enhanced job satisfaction. This highlights a key trait in effective project management: transforming potential disruptions into opportunities.

In the broader context, successful project managers excel in assessing the urgency of interruptions and integrating them into their workflow, without letting the project goals derail. This ability is critical in fast-paced environments where multitasking and adaptability are essential.

\subsection{Interpersonal Skills and Organizational Culture in Project Management}
The integration of interpersonal skills and a supportive organisational culture is essential for successful project management. Project managers need interpersonal skills like empathy, teamwork, and communication to form trusting relationships with their team and handle the demands of many stakeholders. Team dynamics and employee performance are greatly influenced by organisational culture, especially when it comes to the organization's orientation towards individualism or collaboration. Combining these factors shows that a project manager's capacity to successfully negotiate and balance interpersonal relationships within the framework of the dominant organisational culture, creating an atmosphere that encourages teamwork and excellent performance, has a significant impact on the success of the project.

Organizational culture impacts employee behavior and performance. In project management, a culture fostering collaboration and mutual support enhances project outcomes, while a contrary culture can hinder success.


\section{Methods \& Methodology}
\subsection{Approach to Problem}
Summing together various studies, the method for handling project management difficulties is both all-encompassing and flexible. As demonstrated in one study on Agile approaches, this comprehensive approach emphasises the significance of customising project management strategies to meet particular project types, organisational cultures, and employee characteristics. It emphasises how project managers must modify their management approaches to properly deal with time fragmentation and frequent disruptions, demonstrating the importance of flexibility and adaptability in their jobs.

To handle fluctuating market situations, the methods also entails using innovative practises in project and benefits management. This entails determining and closely observing the advantages of technology initiatives as well as comprehending and utilising vital success aspects that are essential to the accomplishment of the project. This puts an empahsis on the efficient resource management and flexibility to control project costs and increase resource efficiency.

Another study evaluates how Project Management Methodologies (PMMs)
has an impact on project's success in industry specific settings like Oil and Gas. Answering important questions on the significance, variety, advantages, and disadvantages of these approaches. This is complemented by empirical results on their impact on project results bringing greater insights.

% The study's methodology integrates knowledge from two primary sources in a qualitative manner. It begins with a thorough analysis of the body of research, which includes both firsthand reports from experienced project managers and several theoretical models that support project management methodologies. Additionally, observational data from project management workshops is incorporated into the analysis. These courses offer a real-world setting where project managers' tactics, attitudes, and methods of operation are explored. The study attempts to develop a thorough knowledge of the non-technical qualities that contribute to successful project management through this integrated qualitative investigation.

\subsection{Analysis Techniques}

To offer a thorough grasp of project management success, the analysis methods used in each study include quantitative and qualitative methods. In order to effectively apply Agile processes, it is necessary to comprehend employee motivations and team dynamics through the application of sociometric and motivation study tools. Benefits management procedures give a framework for analysing the efficacy of project management techniques by concentrating on assessing crucial success elements. Furthermore, a combination of research approaches are utilised, such as semi-structured interviews and cross-sectional questionnaires. Statistical methods such as SPSS are utilised to analyse this data, allowing for a comprehensive assessment of the connection between project success, team dynamics, and project management techniques where applied PMMs had a 32.3\% impact on project's impact. This methodical approach skillfully blends theoretical and practical elements, providing a strong framework for evaluating project management habits and results.

Concurrently, subject analysis is the main method used to identify trends in the methods employed by successful project managers. To find reoccurring themes and patterns in diverse narratives and observations, a synthesised technique is employed during the analysis of qualitative data. A comparative analysis is performed, connecting the found themes with existing frameworks such as the Boy Scouts' Law, to improve validity and depth. This framework functions as a standard by which to evaluate the degree to which the practises of project managers conform to accepted values. Using this dual analytical approach, the research seeks to offer a thorough knowledge of the characteristics that characterise effective project management leadership.



\section{Results Obtained}
\subsection{Conditions}

The empirical based study at the Barcelona Supercomputing Center provides an insight on how Project management methods are applied in research based settings ranging from theoretical to applied research and the traditional project management approaches are applied across the process. This finding aligns with the habits of successful project managers who are adaptable and can handle different project types and complexities. Successful project managers thrive under conditions where they need to balance creativity with productivity, a common scenario in research-based projects as observed in the Barcelona study.

\subsection{Constraints}
This study identifies one major constraints in applying project management methods to research that is the tension between creativity and productivity. In research environments, where creativity is paramount, strict adherence to traditional project management practices is a lot limiting. This highlights the significance that successful project managers need to be empathetic and flexible, and allow more creative freedom and enforce productivity standards as when required.

\subsection{Quality}
This study reveals a dichotomy where scientists at the Barcelona Supercomputing Center use project management methods but only apply them partially. This points to a gap in the quality of project management application in research settings. In contrast, successful project managers, are expected to be knowledgeable and skilled in their methodology, ensuring high-quality project outcomes. The quality of project management is also reflected in how well managers can balance the need for scientific rigor with the flexibility required for innovative research. The ability to adapt project management methodologies to suit the specific nature and goals of a project is a hallmark of quality in successful project management.

\section{Conclusions and Future Works}
\subsection{Suggested Improvements}
Suggestions for future improvements.

\subsection{Limitations to Solution}
Describe scenarions where the solutions dont work well enough.

\subsection{Applications in Real World}
Benefits in real world.

\subsection{Conclusion}
Short summary.

\section{References}
Project management education plays a crucial role in equipping professionals with the necessary skills for successful project execution \cite{pant2008project}.

\cite{kaluzny2022scouting}

\subsection{Appendix}
external references here

\subsection{Acknowledgements}
ACKS

\bibliographystyle{IEEEtran}
\bibliography{references}

\end{document}
