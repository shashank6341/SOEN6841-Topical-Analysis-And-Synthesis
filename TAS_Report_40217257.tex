\documentclass{article}
\usepackage[utf8]{inputenc}
\usepackage{geometry}
\usepackage{graphicx}
\usepackage[hidelinks]{hyperref}

\title{Habits of a Successful Project Manager}
\author{Shashank Verma \\ 40217257
\\
\\
\\
SOEN 6841: Software Project Management}
\date{30/10/2023}


\begin{document}

\maketitle
\pagenumbering{gobble}

\newpage
\tableofcontents
\newpage
\pagenumbering{arabic}

\section{Abstract}
The success of project managers is frequently an amalgam of their technical abilities and interpersonal tendencies. By drawing relationships between traditional management theories and the Scout Law, this paper synthesises the fundamental practises of successful project managers. It investigates the significance of interpersonal skills, proactive problem-solving attitudes, and the ability to sustain strong workplace relationships. A survey of the literature, anecdotal evidence from project management workshops, and the author's own professional observations serve as the foundation for the analysis. The goal of the synthesis is to develop a paradigm that can be used across multiple domains to improve project outcomes and leadership effectiveness.

\section{Introduction}
\subsection{Motivation}
In the current global economy, project managers are widely acknowledged as being essential to the success of projects in a range of businesses. Regardless of the business, their strong management abilities are crucial to raising project productivity and efficiency. Despite the advancements in project management tools and processes, the human element—exemplified by the behaviours of project managers—remains a crucial component of project success. Project management is changing, and this shows how important it is to value human skills in addition to technological knowledge. Project management today includes human dynamics, teamwork, and interpersonal communication in addition to technical execution. Project management expertise is becoming more and more in demand as organisations embrace project-centric approaches. \cite{pant2008project}

Now more than ever, educational institutions—especially universities—need to adapt to these changes in the business. It is imperative that project management courses be modified to provide a more comprehensive set of skills. This adaptation is a proactive strategy to equip upcoming project managers for the complex needs of a changing work environment, rather than only a reaction to market demands. It is imperative that technical and interpersonal abilities are balanced in project management education, which highlights the need for a paradigm change in the way project management is taught and perceived.

\subsection{Problem Statement}
The primary issue with present project management is the overemphasis on technical abilities, which frequently ignores soft or people skills. Frameworks such as the PMBOK Guide, which emphasise technical skills over soft skills runs the danger of ignoring soft skills that are just as important—like team dynamics, leadership, and communication—also exhibit this mismatch. This argues for a more integrated approach to project management education, emphasising both technical and interpersonal abilities equally. The objective is to produce managers who excel in both the technical and human aspects of project management by matching academic training with industrial needs. This well-rounded approach is essential for producing project managers who can successfully lead projects in the diverse and changing professional environment of today. \cite{pant2008project}



\subsection{Objectives}
This case study aims to identify the non-technical capabilities that are essential to project managers' success. By doing this, we seek to enhance project leadership development and training within organisations. By emphasising the significance of soft skills in achieving project success, this study offers practitioners and their teams practical insights that can be utilised to improve project managers' abilities.

\section{Background Material}
\subsection{The Scout Law as a Leadership Model}

Effective leadership styles in project management significantly impact project success. Project leaders must not only coordinate tasks but also inspire and guide their teams. Unfortunately, project management education often prioritizes technical skills over leadership development. Studies on scouting reveal the importance of a balanced leadership approach. Scouting's focus on individual growth within teams aligns with project management requirements like adaptability, decision-making, problem-solving, and ethical judgment. Scouting's emphasis on real-world challenges and teamwork parallels project management needs. Insights from scouting studies show how young individuals develop strategic thinking, communication, and ethical leadership—critical competencies for project managers. Combining technical expertise with leadership skills fosters a collaborative environment vital for achieving collective project goals. Project management education must evolve to incorporate these leadership aspects, drawing from scouting methodologies to better prepare future professionals for project success. \cite{kaluzny2022scouting}

\subsection{Adaptability in the Face of Disruption}
In the realm of successful project management, the capability to manage and adapt to interruptions is pivotal. Few studies from engineering consultancy exemplifies this, showing how a project manager navigated a work environment rife with interruptions, varying from sudden phone calls to scheduled meetings. Interestingly, 43.2\% of these interruptions were beneficial, contributing to goal achievement and enhanced job satisfaction. This highlights a key trait in effective project management: transforming potential disruptions into opportunities.

In the broader context, successful project managers excel in assessing the urgency of interruptions and integrating them into their workflow, without letting the project goals derail. This ability is critical in fast-paced environments where multitasking and adaptability are essential. \cite{mordu2016managing}

\subsection{Interpersonal Skills and Organizational Culture in Project Management}
The integration of interpersonal skills and a supportive organisational culture is essential for successful project management. Project managers need interpersonal skills like empathy, teamwork, and communication to form trusting relationships with their team and handle the demands of many stakeholders. Team dynamics and employee performance are greatly influenced by organisational culture, especially when it comes to the organization's orientation towards individualism or collaboration. Combining these factors shows that a project manager's capacity to successfully negotiate and balance interpersonal relationships within the framework of the dominant organisational culture, creating an atmosphere that encourages teamwork and excellent performance, has a significant impact on the success of the project. \cite{nusari2018impact}

Organizational culture impacts employee behavior and performance. In project management, a culture fostering collaboration and mutual support enhances project outcomes, while a contrary culture can hinder success.


\section{Methods \& Methodology}

\subsection{Methodological Framework for Investigating Project Manager Habits}

\subsubsection{Literature Review}

An comprehensive literature analysis was performed using Google Scholar to investigate the habits of successful project managers. The study focused on terms such as "project manager traits," "successful project management habits," and "leadership in project management." The selection criteria was based on publications' relevance, and recentness. A few important studies were "What is a good project manager?" \cite{bredillet2015good}, "Leadership competency profiles of successful project managers" emphasise critical thinking and emotional intelligence \cite{muller2010leadership}; "Patterns in the project managers’ rhythms, habits, routines and rituals" explore daily routines and habits \cite{sigurdhssonpatterns}; and "Project managers’ and change managers’ contribution to success", analyses the roles of project and change management in project success \cite{pollack2016project}. These resources provide an overall perspective on the good habits of a project manager.

\subsubsection{Identification of Key Traits and Practices} 

The key traits and practices of successful project managers identified from the literature review align well with the attributes in the original case study. These include competence in both attribute and performance based dimensions, emphasizing intellectual, managerial, and emotional competencies like critical thinking and emotional intelligence \cite{muller2010leadership}. The importance of organization, communication, and continuous improvement in daily routines \cite{sigurdhssonpatterns}, along with skills in managing costs and navigating organizational change, echoes the workshop's findings on traits like sociability, respectfulness, positive demeanor, and the ability to 'speak truth to power' \cite{bredillet2015good}. This synthesis of literature underscores the consistent themes in defining effective project management practices.

\subsubsection{Inclusion of Diverse Perspectives}

In addressing the problem of identifying successful project management habits, the approach included seeking diverse perspectives from various industries, organizational sizes, and cultural contexts. This was achieved through the analysis of the provided case studies. Each study brought a unique angle: one examined the philosophical underpinnings of what constitutes competence in project management \cite{bredillet2015good}, another focused on leadership competencies in different types of projects \cite{muller2010leadership}, while a third explored the daily habits and routines of project managers \cite{sigurdhssonpatterns}. Additionally, the relationship between project management and change management was considered, highlighting the adaptability of project management practices across different organizational changes \cite{pollack2016project}. This multifaceted approach ensured a comprehensive understanding of project management habits, encompassing a wide range of scenarios and challenges faced in different environments and cultures.

\subsection{Analytical Methods in Deciphering Project Management Traits}

\subsubsection{Thematic Analysis}

The procedure involved systematically using a qualitative analysis approach to classify and identify recurring themes and patterns linked to effective project management practices. By categorizing key characteristics, behaviors, and competencies, the analysis uncovered reoccurring themes. These themes, reflecting the diversity of project management roles across various industries and organizational sizes, included critical thinking, emotional intelligence, effective communication, and adaptability. This approach enabled a deeper insight into the essential routines and habits that are crucial for success in project management, as corroborated by numerous studies.


\subsubsection{Comparative Analysis}

In analyzing the results, a comparative approach was used to discern how factors like project size, industry, or manager experience level influence project management habits. The analysis revealed variations in competencies across project types, showing the impact of industry and complexity \cite{muller2010leadership}. It also highlighted how managerial experience affects the adoption of certain habits \cite{sigurdhssonpatterns} and how organizational context shapes project management approaches, particularly in the interplay between project and change management \cite{pollack2016project}. This method offered a deeper understanding of the varied influences on effective project management practices.

\subsubsection{Synthesis of Findings}

The synthesis of findings from various case studies reveals a multifaceted profile of successful project management, emphasizing the integration of both intellectual and emotional competencies underpinned by strong ethical considerations \cite{bredillet2015good}. Key attributes identified include critical thinking, influence, and motivation \cite{muller2010leadership}. Daily habits and routines of project managers, such as effective organization and continuous improvement, align closely with these competencies, underscoring their practical application in day-to-day project management \cite{sigurdhssonpatterns}. Additionally, the interaction between project management and change management highlights the necessity for project managers to adeptly navigate organizational changes, with competencies complementing each role in effective project execution \cite{pollack2016project}. This synthesis also reconciles differing perspectives, such as the balance between soft skills and technical expertise, and consistently points to universally acknowledged traits like communication, leadership, and adaptability as central to the effectiveness of project managers.


\subsection{Investigating the Key Traits in Project Management}
Project management relies on both interpersonal and cognitive skills. The traits such as empathy and self-awareness affect staff morale and ultimately determine whether projects succeed or not. A good project manager is skilled at team leadership, negotiations, and motivation towards higher team cohesion and efficiency. The other essential skills that make up critical thinking and creative problem solving include navigating through complexity, making good choices, and identifying potential solutions. It is necessary because, besides dealing with time flexibly, it enables one to manage rapidly changing events that occur in project work. Project managers need to keep on learning and keep up with changes within their fields. Using the LDQ and other tools show how essential those traits are for success and that any research based on them is credible.\cite{muller2010leadership}

\subsection{Analysis of Project Management Practices}
\begin{figure}[htp]
    \centering
    \includegraphics[width=10cm]{SuccessPM}
    \caption{Successful vs Failure Project Management}
    \label{fig:SuccessPM}
\end{figure}
The good project managers combine the knowledge about changes with conventional task managing practices and manage adjusting them individually for each issue’s specificity, thus succeeding. They are able to foresee possible risks in projects environments and also engage the different parties effectively involved in projects. Adaptability along with effective internal communications and collaborative work culture plays a critical role in designing an appropriate leadership style that meets the situation and requirements of the team. In addition, they acquire new knowledge on current affairs related to project and change management for implementing successful projects. In conclusion, their success depends on an organized effort on risk analysis, stakeholders involvement, adaptation ability and communication customized for each project need.\cite{pollack2016project}

\subsection{Comparative Studies}
Project managers who succeed do so because they take the time to plan each part of their project as well as risk management and contingency plans. This is where they do very well at. They are able to get projections right, time lining, budgeting etc so that the project complies with available time frames. Systematic approach allows them to take appropriate measures and maintain deliverables’ quality regardless of new requirements. Expertise lies in their tendency to stay ahead by continuously monitoring risks, identifying potential issues that could delay operations or cause disruption, and implementing mitigation measures. They always keep track of the project progress and address quickly all the deviation with corrective action. Effective communication backed with stakeholder involvement as a solid management policy for handling all issues involved of software project management that will help to avoid failure which is the lot of many other managers.\cite{jones2004software}

\subsection{}


% \subsection{Approach to Problem}
% Summing together various studies, the method for handling project management difficulties is both all-encompassing and flexible. As demonstrated in one study on Agile approaches, this comprehensive approach emphasises the significance of customising project management strategies to meet particular project types, organisational cultures, and employee characteristics. It emphasises how project managers must modify their management approaches to properly deal with time fragmentation and frequent disruptions, demonstrating the importance of flexibility and adaptability in their jobs.

% To handle fluctuating market situations, the methods also entails using innovative practises in project and benefits management. This entails determining and closely observing the advantages of technology initiatives as well as comprehending and utilising vital success aspects that are essential to the accomplishment of the project. This puts an empahsis on the efficient resource management and flexibility to control project costs and increase resource efficiency.

% Another study evaluates how Project Management Methodologies (PMMs)
% has an impact on project's success in industry specific settings like Oil and Gas. Answering important questions on the significance, variety, advantages, and disadvantages of these approaches. This is complemented by empirical results on their impact on project results bringing greater insights.

% % The study's methodology integrates knowledge from two primary sources in a qualitative manner. It begins with a thorough analysis of the body of research, which includes both firsthand reports from experienced project managers and several theoretical models that support project management methodologies. Additionally, observational data from project management workshops is incorporated into the analysis. These courses offer a real-world setting where project managers' tactics, attitudes, and methods of operation are explored. The study attempts to develop a thorough knowledge of the non-technical qualities that contribute to successful project management through this integrated qualitative investigation.

% \subsection{Analysis Techniques}

% To offer a thorough grasp of project management success, the analysis methods used in each study include quantitative and qualitative methods. In order to effectively apply Agile processes, it is necessary to comprehend employee motivations and team dynamics through the application of sociometric and motivation study tools. Benefits management procedures give a framework for analysing the efficacy of project management techniques by concentrating on assessing crucial success elements. Furthermore, a combination of research approaches are utilised, such as semi-structured interviews and cross-sectional questionnaires. Statistical methods such as SPSS are utilised to analyse this data, allowing for a comprehensive assessment of the connection between project success, team dynamics, and project management techniques where applied PMMs had a 32.3\% impact on project's impact. This methodical approach skillfully blends theoretical and practical elements, providing a strong framework for evaluating project management habits and results.

% Concurrently, subject analysis is the main method used to identify trends in the methods employed by successful project managers. To find reoccurring themes and patterns in diverse narratives and observations, a synthesised technique is employed during the analysis of qualitative data. A comparative analysis is performed, connecting the found themes with existing frameworks such as the Boy Scouts' Law, to improve validity and depth. This framework functions as a standard by which to evaluate the degree to which the practises of project managers conform to accepted values. Using this dual analytical approach, the research seeks to offer a thorough knowledge of the characteristics that characterise effective project management leadership.



\section{Results Obtained}
\subsection{Conditions}

% Successful project managers thrive under conditions that foster a balance between ethical actions and practical competencies. These managers excel in environments that recognize the importance of both deontological (duty-focused) and consequentialist (outcome-focused) approaches. The integration of attribute-based and performance-based competencies, highlighted in the case study, aligns with the traits identified in your original study, such as reliability, team care, and the ability to follow through and manage costs effectively. Additionally, the emphasis on praxis (action) and phronesis (practical wisdom) in complex and uncertain project environments resonates with the need for project managers to be adaptable, understanding, and capable of speaking truth to power. This approach underscores the importance of balancing theoretical knowledge with practical application, ensuring that project managers are not just implementers of best practices but also critical thinkers and problem solvers who are sociable, respectful, and capable of leading their teams effectively in diverse circumstances.

% The effectiveness of project management attributes is highly contingent upon the specific project environment, implying that flexible and adaptive project managers are essential for success. Characteristics like strategic vision, good communication, and adaptability are critical in larger organisations and more complicated initiatives. These characteristics aid in navigating intricate project structures and stakeholder landscapes, guaranteeing thorough management and alignment with more general organisational objectives.

% On the other side, emphasis is placed on greater practical project involvement, effective interpersonal skills, and a customised leadership style in smaller projects or organisations. In this case, open communication and close teamwork are crucial because they create a project atmosphere that is more personal and responsive.

% Furthermore, a crucial factor in influencing the efficacy of project management attributes is the organisational structure. Because of the particular difficulties and dynamics of matrix or projectized organisations, expertise in resource negotiation, handling conflict, and leading cross-functional teams becomes increasingly important.

% Furthermore, the characteristics necessary for efficient project management also depend on the stage of the project's lifespan. For example, careful planning and risk assessment are necessary during the initiation and planning stages, but quality control, team morale, and stakeholder satisfaction are the main priorities during the execution and closing stages.

% Essentially, there is no one-size-fits-all method for successful project management; rather, it necessitates a sophisticated comprehension of a range of variables, including organisational size, project complexity, team dynamics, and the particular stage of the project lifecycle. Project managers will be able to modify their tactics and routines to best meet the unique requirements and difficulties of every project environment thanks to this dynamic approach.\cite{hyvari2006success}

% -----

The effectiveness of project managers significantly depends on the specific conditions of their projects, necessitating adaptability and a nuanced approach to management. Balancing ethical and practical competencies is crucial, incorporating both duty-focused and outcome-focused strategies. This balance is augmented by the integration of praxis and phronesis \cite{bredillet2015good}, essential in complex project environments. Leadership skills such as strategic vision, adaptability, and effective communication are especially vital in larger organizations, where they help in navigating intricate structures and aligning projects with broader organizational goals. In contrast, smaller projects or organizations require more hands-on involvement \cite{sigurdhssonpatterns}, emphasizing interpersonal skills and a personalized leadership style to foster a more intimate and responsive project atmosphere \cite{pollack2016project}.

The project's organizational structure also influences the efficacy of project management traits. In matrix or projectized environments, expertise in resource negotiation, conflict management, and leading cross-functional teams is paramount. Additionally, the stage of the project's life cycle plays a crucial role in determining necessary management attributes. Early stages demand thorough planning and risk assessment, while later stages focus on quality control, team morale, and stakeholder satisfaction. Ultimately, successful project management is not a one-size-fits-all solution but requires understanding and adapting to various factors, including organizational size, project complexity, and team dynamics. This adaptive approach allows project managers to modify their tactics and routines to best meet the unique requirements of each project environment \cite{hyvari2006success}.

\subsection{Constraints}
% This study identifies one major constraints in applying project management methods to research that is the tension between creativity and productivity. In research environments, where creativity is paramount, strict adherence to traditional project management practices is a lot limiting. This highlights the significance that successful project managers need to be empathetic and flexible, and allow more creative freedom and enforce productivity standards as when required.

% Within the field of project management, the effectiveness of different attributes and methodologies is not universally applicable; rather, it differs greatly depending on the unique conditions of every project. Which project management attributes work best depends on a number of important aspects, including the project's size and kind, organisational structure, and external environmental conditions. 

% The most effective project managers are those who are flexible and adaptive, able to modify their methods and style of management to suit the particular requirements of various projects. In the quickly changing business contexts of today, where outside variables like market dynamics and technology breakthroughs can have a big impact on project outcomes, this adaptability is especially crucial.

% Furthermore, a key component of successful project management is the capacity to recognise which behaviours and characteristics to employ in different situations. This entails being aware of the subtleties of organisational structure and culture, which have a big influence on how well some management techniques work.

% Fundamentally, the dynamic and diverse character of projects demands an adaptive approach to project management, wherein the manager's success frequently rests on his or her capacity to adjust and react to a variety of various and changing project contexts.\cite{markopoulos2005project}

The effectiveness of project management traits and methodologies varies significantly based on each project's unique context. Leadership competencies, for example, differ across project types, with their effectiveness contingent on factors like project complexity, strategic importance, and contract type, suggesting that certain leadership traits may not be universally applicable in every project environment\cite{muller2010leadership}. Additionally, there is variability in project managers'  habits, routines, and rituals, indicating the necessity for adaptation to both individual and organizational contexts \cite{sigurdhssonpatterns}.

Furthermore, successful project management requires adaptability and the capability to modify management styles to fit the specific needs of different projects \cite{markopoulos2005project}. This adaptability is particularly critical in fast-evolving business environments where external factors such as market dynamics and technological advancements can significantly influence project outcomes. Recognizing which behaviors and traits to employ in varying situations is vital, necessitating awareness of the intricacies of organizational structure and culture that significantly influence the efficacy of certain management techniques. The dynamic and diverse nature of projects demands an adaptive approach to project management, where a manager's success often hinges on their ability to adjust and react to various and evolving project contexts \cite{pollack2016project}.

\subsection{Quality}
% This study reveals a dichotomy where scientists at the Barcelona Supercomputing Center use project management methods but only apply them partially. This points to a gap in the quality of project management application in research settings. In contrast, successful project managers, are expected to be knowledgeable and skilled in their methodology, ensuring high-quality project outcomes. The quality of project management is also reflected in how well managers can balance the need for scientific rigor with the flexibility required for innovative research. The ability to adapt project management methodologies to suit the specific nature and goals of a project is a hallmark of quality in successful project management.

Certain characteristics and methods stick out in project management as being essential to a project's success. They consist of stakeholder engagement, risk management, strategic planning, and effective communication. These practises are not only advantageous, but they also work well in a wide range of project kinds and situations. Crucial project success measures including budget adherence, stakeholder satisfaction, and completion time are all directly and significantly impacted by them.

Strategic planning aids in negotiating challenging project environments and successfully accomplishing goals, while effective communication guarantees that all stakeholders are in agreement with the project's aims and advancement. Stakeholder engagement is critical to retaining the support and satisfaction of all stakeholders engaged in a project, and risk management is critical to identifying and addressing possible concerns that could derail it.\cite{fernandes2013identifying}


\section{Conclusions and Future Works}
\subsection{Suggested Improvements}
\begin{figure}[htp]
    \centering
    \includegraphics[width=10cm]{Suggested Improvements}
    \caption{Framework for Project Management Research}
    \label{fig:Suggested Improvements}
\end{figure}
There are a number of crucial areas in the field of project management where more study and development could significantly increase the efficiency of project managers. It is essential to comprehend and manage project complexity well, requiring for the creation of new frameworks and tools. Furthermore, as they have a big influence on project success, the social dynamics of project management—including stakeholder engagement, team dynamics, and interpersonal relationships—should also receive a lot of attention.

Aligning project goals with organisational objectives and stakeholder values is crucial for value generation, which is another crucial factor. This calls for a more thorough comprehension of how to incorporate broader company goals into the design and implementation of projects. Furthermore, the early phases of project conception are critical, emphasising the necessity for excellent project planning and strategy abilities right away.

Moreover, a crucial topic is the advancement of project managers themselves. There is an increasing demand for training that goes beyond merely technical abilities and includes critical thinking, flexibility, and reflective practises. Project managers must make this change in order to lead and negotiate the more complicated and dynamic environments of contemporary projects.

Taken as a whole, these areas are crucial for further development in the field of project management, guaranteeing that managers have the abilities and know-how to successfully oversee and steer projects in a variety of settings.\cite{winter2006directions}


\subsection{Limitations to Solution}

% Describe scenarions where the solutions dont work well enough

Within the realm of project management, organisational limitations as well as the personal qualities of the project manager can have a big impact on how effective common practises and traits are. Ineffective leadership can undermine a project's effectiveness due to negative personal traits such poor communication, authority misuse, or inexperience. These personal constraints serve as a reminder of the value of interpersonal and leadership abilities in addition to technical expertise in project management.

Organisational limitations, such as little resources, poor planning, and a lack of backing from upper management, can present serious difficulties. These elements can limit a project manager's capacity to carry out best practises and accomplish project objectives, highlighting the challenges associated with using conventional project management techniques in a variety of organisational contexts.

These observations highlight how important flexibility and adaptation are to project management. Project managers must possess a variety of abilities and be able to handle the different obstacles that arise in distinct project environments. It is imperative that project managers devise techniques that are resilient and flexible, able to tackle intricate situations and transcend individual and institutional constraints.\cite{toor2009ineffective}

\subsection{Applications in Real World}
% Benefits in real world.
The application of methodologies like PERT and CPM in various industries\cite{cicmil2006rethinking}, showcases the importance of adaptability, strategic planning, and data-driven decision-making in project management. These methodologies are pivotal in optimizing project duration and cost, highlighting the need for effective time management and resource allocation in complex projects with interdependent tasks. The versatility of these methodologies underscores their utility in meeting the unique demands of different projects, regardless of scale or type, and emphasizes the value of data-driven approaches for informed decision-making.

In contrast\cite{knudson2009software}, discusses the nuances of applying project management in real-world industry projects compared to academic settings. Real-world projects, characterized by their dynamic nature, necessitate flexibility, real-time problem-solving, and adaptability. This is due to the ever-changing stakeholder expectations, shifting project scopes, and the balance required between technical demands and business objectives. This real-world context demands a practical application of project management principles, where theoretical knowledge must be adapted to fluid and unpredictable scenarios.

In summary, successful project management in real-world settings requires a blend of theoretical knowledge and practical adaptability. Methodologies like PERT and CPM demonstrate the importance of strategic planning and data-driven decision-making across various industries, while the dynamic nature of real-world projects calls for flexibility and real-time problem-solving. This combination of traits ensures that project managers can effectively navigate the complexities and unique challenges presented in different project environments.

\subsection{Conclusion}
% Short summary.

To handle complicated projects, effective project management requires an all-encompassing and flexible strategy that integrates cutting-edge approaches and tools. With an emphasis on the social dimensions of project management, it's vital to match organisational goals with project goals and develop managers' critical thinking and adaptability skills. Project managers must possess a variety of skills and flexibility in order to efficiently traverse different project contexts and overcome personal and organisational problems. The practical use of techniques such as PERT and CPM highlights the significance of strategic planning and data-informed decision-making. At the end of the day, effective project management is about combining academic understanding with real-world flexibility, which means that ongoing education and development are required to stay up with the ever-changing project environment.



% \section{References}


% \subsection{Appendix}
% external references here

% \subsection{Acknowledgements}
% ACKS

\bibliographystyle{IEEEtran}
\bibliography{references}

\end{document}
