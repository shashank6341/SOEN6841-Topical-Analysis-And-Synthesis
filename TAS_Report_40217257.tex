\documentclass{article}
\usepackage[utf8]{inputenc}
\usepackage{geometry}
\usepackage{graphicx}
\usepackage{hyperref}

\title{Habits of a Successful Project Manager}
\author{Shashank Verma \\ 40217257}
\date{30/10/2023}


\begin{document}

\maketitle
\pagenumbering{gobble}

\newpage
\tableofcontents
\newpage
\pagenumbering{arabic}

\section{Abstract}
The success of project managers is frequently an amalgam of their technical abilities and interpersonal tendencies. By drawing relationships between traditional management theories and the Scout Law, this paper synthesises the fundamental practises of successful project managers. It investigates the significance of interpersonal skills, proactive problem-solving attitudes, and the ability to sustain strong workplace relationships. A survey of the literature, anecdotal evidence from project management workshops, and the author's own professional observations serve as the foundation for the analysis. The goal of the synthesis is to develop a paradigm that can be used across multiple domains to improve project outcomes and leadership effectiveness.

\section{Introduction}
\subsection{Motivation}
Two primary factors contribute to the motivation for this research. First, it is widely acknowledged that project managers are critical to project success. Their capacity to manage effectively can significantly improve the productivity and efficiency of these initiatives, regardless of industry. Second, despite advancements in project management tools and methodologies, the human element's value stays constant. The way project managers lead and conduct is essential to the success of their projects. This research attempts to look into these crucial areas—the impact of project management and project manager behaviors—in order to increase project success rates across various sectors.


\subsection{Problem Statement}
The purpose of this research is to investigate the essential habits and characteristics that distinguish successful project managers. While technical abilities are typically emphasised, this study highlights the importance of people skills and good interpersonal interaction, which are as important but are often overlooked. It is to determine which habits are most significant for project managers and how they relate to the broader practises of good leadership and management.


\subsection{Objectives}
This case study aims to identify the non-technical capabilities that are essential to project managers' success. By doing this, we seek to enhance project leadership development and training within organisations. By emphasising the significance of soft skills in achieving project success, this study offers practitioners and their teams practical insights that can be utilised to improve project managers' abilities.
\section{Background Material}
\subsection{Subject 1}
Details about subject 1.

\subsection{Subject 2}
Details about subject 2.

% Add more subsections as needed

\section{Methods \& Methodology}
\subsection{Approach}
Discussion on how the problem was approached.

\subsection{Techniques Used}
techniques used for result anaylsis.

\section{Results Obtained}
\subsection{Conditions}
Under what conditions were these results obtaineD?

\subsection{Constraints}
constraints on the report.

\subsection{Quality}
How is the quality of the case study.

\section{Conclusions and Future Works}
\subsection{Suggested Improvements}
Suggestions for future improvements.

\subsection{Limitations to Solution}
Describe scenarions where the solutions dont work well enough.

\subsection{Applications in Real World}
Benefits in real world.

\subsection{Conclusion}
Short summary.

\section{References}
Use the \textbackslash cite command to insert citations. For example, see \cite{reference1}.

\subsection{Appendix}
external references here

\subsection{Acknowledgements}
ACKS

\bibliographystyle{IEEEtran}
\bibliography{references}

\end{document}
