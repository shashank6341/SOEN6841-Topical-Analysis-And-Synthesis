\documentclass{article}
\usepackage[utf8]{inputenc}
\usepackage{geometry}
\usepackage{graphicx}
\usepackage{hyperref}

\title{Habits of a Successful Project Manager}
\author{Shashank Verma \\ 40217257}
\date{30/10/2023}


\begin{document}

\maketitle
\pagenumbering{gobble}

\newpage
\tableofcontents
\newpage
\pagenumbering{arabic}

\section{Abstract}
The success of project managers is frequently an amalgam of their technical abilities and interpersonal tendencies. By drawing relationships between traditional management theories and the Scout Law, this paper synthesises the fundamental practises of successful project managers. It investigates the significance of interpersonal skills, proactive problem-solving attitudes, and the ability to sustain strong workplace relationships. A survey of the literature, anecdotal evidence from project management workshops, and the author's own professional observations serve as the foundation for the analysis. The goal of the synthesis is to develop a paradigm that can be used across multiple domains to improve project outcomes and leadership effectiveness.

\section{Introduction}
\subsection{Motivation}
Two primary factors contribute to the motivation for this research. First, it is widely acknowledged that project managers are critical to project success. Their capacity to manage effectively can significantly improve the productivity and efficiency of these initiatives, regardless of industry. Second, despite advancements in project management tools and methodologies, the human element's value stays constant. The way project managers lead and conduct is essential to the success of their projects. This research attempts to look into these crucial areas—the impact of project management and project manager behaviors—in order to increase project success rates across various sectors.


\subsection{Problem Statement}
The purpose of this research is to investigate the essential habits and characteristics that distinguish successful project managers. While technical abilities are typically emphasised, this study highlights the importance of people skills and good interpersonal interaction, which are as important but are often overlooked. It is to determine which habits are most significant for project managers and how they relate to the broader practises of good leadership and management.


\subsection{Objectives}
This case study aims to identify the non-technical capabilities that are essential to project managers' success. By doing this, we seek to enhance project leadership development and training within organisations. By emphasising the significance of soft skills in achieving project success, this study offers practitioners and their teams practical insights that can be utilised to improve project managers' abilities.

\section{Background Material}
\subsection{People Orientation and Interpersonal Dynamics}
This part of the study emphasises the significance interpersonal skills are to effective project leadership. We seek to explore the impact of a people-oriented approach on project success by looking at how project managers interact with teams and stakeholders.


\subsection{Adaptability in the Face of Disruption}
The study delves into how effective project managers handle delays and broken schedules. This entails investigating how they efficiently manage their time and keep control of projects in spite of repeated setbacks.

\subsection{Project Management Working Styles}
The inner workings of a project manager's approach to work are explored, with an emphasis on having the ability to multitask and communicate openly. The purpose of the study is to determine how these working methods impact the finished outcomes of their projects.

\subsection{Fundamental Project Management Responsibilities}
The main activities that project managers need to manage are examined, including their capacity to fulfil assignments, care for and aid the individuals they manage, as well as effective cost management.

\subsection{The Scout Law as a Leadership Model}
A noteworthy section of the research draws similarities between the Scout Law's tenets and the behaviours of successful project managers. The objective of this comparison is to identify a possible global norm for project management success and leadership.

\section{Methods \& Methodology}
\subsection{Approach to Problem}
The study's methodology integrates knowledge from two primary sources in a qualitative manner. It begins with a thorough analysis of the body of research, which includes both firsthand reports from experienced project managers and several theoretical models that support project management methodologies. Additionally, observational data from project management workshops is incorporated into the analysis. These courses offer a real-world setting where project managers' tactics, attitudes, and methods of operation are explored. The study attempts to develop a thorough knowledge of the non-technical qualities that contribute to successful project management through this integrated qualitative investigation.

\subsection{Analysis Techniques}
The study's main approach for discovering patterns in the actions and routines of effective project managers is subject analysis. Using a synthesised approach, qualitative data is analysed to identify themes and patterns that appear in various narratives and observations. In order to provide validity and depth, the study also conducts a comparative analysis, placing the themes that have been identified against pre-existing frameworks. The Boy Scouts' Law is one such framework that is utilised for this comparison analysis. It acts as a standard to assess how well the project managers' practises adhere to these time-tested values. The research aims to offer a comprehensive knowledge of the characteristics that characterise effective project management leadership through the use of this dual analytical method.

\section{Results Obtained}
\subsection{Conditions}
Under what conditions were these results obtaineD?

\subsection{Constraints}
constraints on the report.

\subsection{Quality}
How is the quality of the case study.

\section{Conclusions and Future Works}
\subsection{Suggested Improvements}
Suggestions for future improvements.

\subsection{Limitations to Solution}
Describe scenarions where the solutions dont work well enough.

\subsection{Applications in Real World}
Benefits in real world.

\subsection{Conclusion}
Short summary.

\section{References}
Use the \textbackslash cite command to insert citations. For example, see \cite{reference1}.

\subsection{Appendix}
external references here

\subsection{Acknowledgements}
ACKS

\bibliographystyle{IEEEtran}
\bibliography{references}

\end{document}
